\documentclass{amsart}
\usepackage{amssymb,amsfonts,amsmath}
\usepackage[alphabetic,y2k,lite]{amsrefs}
\usepackage{bbm}
\usepackage{graphicx}
\usepackage{fullpage, mystyle}
\usepackage{tikz-cd}


\newcommand{\torus}{{\mathbb{T}^2}}

\newcommand{\RRR}{{\underline{\mathfrak{R}}}}
\newcommand{\QQQ}{{\underline{\mathfrak{Q}}}}
\newcommand{\CCC}{{\underline{\mathfrak{C}}}}
\newcommand{\PPP}{{\underline{\mathbf{\Phi}}}}
\newcommand{\TTT}{{\underline{\mathbf{\Theta}}}}
\newcommand{\LLL}{{\underline{\mathfrak{L}}}}

\newcommand{\cev}[1]{\overset{\leftarrow}{#1}}


%% for allowing other symbols to be arrow in tikzcd
%% e.g. A \ar[r, symbol=\subseteq] & B
\tikzset{
  symbol/.style={
    draw=none,
    every to/.append style={
      edge node={node [sloped, allow upside down, auto=false]{$#1$}}}
  }
}

\begin{document}


\title{Variation on a Variational Principle}
\author{Alice Kwon}
   \address{TODO SUNY Maritime address}
    \email{TODO email}
    \urladdr{TODO webpage}
\author{Ying Hong Tham}
   \address{Department of Mathematics, Stony Brook University, 
            Stony Brook, NY 11794, USA}
    \email{yinghong.tham@stonybrook.edu}
    \urladdr{http://www.math.sunysb.edu/\textasciitilde yinghong/}

\begin{abstract}
We prove that alternating links in the thickened torus remaini hyperbolic
after certain augmentations.
\end{abstract}
\maketitle

%===============================================================================
\section{Introduction}

\textbf{Definitions, Notations, and Conventions}

Always assume surface $\Sigma$ (our case $\Sigma = \torus$) is oriented.

define cellular decomp of surface

$V,E,F$ are set of vertices, edges, faces.

$\vec{E}$ is the set of oriented edges.
We may identify an oriented edge $\vec{e}$ with the pair $(f_{\vec{e}}, e)$,
where $f_{\vec{e}}$ is the face to the left of $\vec{e}$.

%===============================================================================

\section{Set up?}

Recall circle pattern.

A circle pattern is determined by the radius of the circle $C_f$ associated to each face, $r_f$,
and the angle that each edge subtends in adjacent faces, $\vphi_{\vec{e}}$. (see figure TODO)
Thus determines, and is determined by a point in $\RRR \times \QQQ$, where

\begin{itemize}
	\item $\RRR := \RR_+^F = \{(r_f)_{f\in F} | r_f \in \RR_+\}$
	\item $\QQQ := \RR^{\vec{E}} =
		\{(\vphi_{\vec{e}})_{\vec{e} \in \vec{E}} | \vphi_{\vec{e}} \in \RR\}$
\end{itemize}
but clearly not every point $c\in \RRR \times \QQQ$ determines a circle pattern.


On $\RRR \times \QQQ$, there are several functions to consider:
\begin{itemize}
	\item $\Phi_f = 2 \sum_{\vec{e} \in \del f} \vphi_{\vec{e}}$, measuring the cone angle
		at the center of $C_f$
	\item $\theta_e = \pi - \vphi_{\vec{e}} - \vphi_{\cev{e}}$
	\item $l_{\vec{e}} = r_{f_{\vec{e}}} \sin \vphi_{\vec{e}}$
\end{itemize}

These fit together to give maps to the following spaces:

\begin{itemize}
	\item $\PPP := \RR^F = \{(\Phi_f)_{f\in F} | \Phi_f \in \RR\}$
	\item $\TTT := \RR^E = \{(\theta_e)_{e \in E} | \theta_e \in \RR\}$
	\item $\LLL := \RR^{\vec{E}} = \{(l_{\vec{e}})_{\vec{e} \in E} | l_{\vec{e}} \in \RR\}$
\end{itemize}


Our main argument is to deform "degenerate" circle patterns, where adjacent circles may
be identical or tangent, into ones that don't look so degenerate,
hence it is conveneint to extend the notion of circle pattern:

\begin{definition}
An \emph{extended circle pattern} is $c = ((r_f), (\vphi_{\vec{e}})) \in \RRR \times \QQQ$
such that $l_{\vec{e}} = l_{\cev{e}}$ for all edges $e\in E$. We denote by
$\CCC$ the set of extended circle patterns.
\end{definition}


\begin{tikzcd}
\CCC = \{l_{\vec{e}} = l_{\cev{e}} \} \ar[r,symbol=\subseteq]
	& \RRR \times \QQQ \ar[r] \ar[d]
	&\TTT \\
	& \PPP
\end{tikzcd}


[Check this] The usual notion of circle pattern would be restricted to those $c\in \CCC$
with $\vphi_{\vec{e}} \in (0,\pi)$ and $\theta_e \in (0,\pi)$.
One may consider deforming a circle pattern so as to have some $\theta_e$ approach 0 or $\pi$.
The limit $\theta_e \to \pi$ is easy to picture, one simply gets that the two circles
$C_f, C_{f'}$ of the adjacent faces become tangent. The limit $\theta_e \to 0$
is a bit more complex, as the final shape of $Q_e$ (the quadrangle associated to $e$)
depends also on $\vphi_{\vec{e}}$ (which equals $\pi - \vphi_{\cev{e}}$ in the limit.
If we parametrize circle patterns by $(r_f)$ and $(\theta_e)$ as in TODO \ocite{BS},
the limiting shape would depend on how $r_f$ approaches $r_{f'}$,
some sort of blow up stuff TODO.


We will mostly be working with extended circle patterns that mostly ``look normal",
with all $\vphi_{\vec{e}}$ in some range $(-\veps, \pi + \veps)$.

\begin{definition}
An extended circle pattern is said to be \emph{face non-singular}
if $\Phi_f = 2\pi$ for all faces $f$; it is said to be \emph{vertex non-singular}
if $\sum_{e \ni v} \theta_e = 2\pi$ for all vertices $v$.
Finally it is said to be \emph{non-singular} if it is both.
\end{definition}

non-degenerate circle pattern?


\begin{definition}
Given some extended circle pattern $c \in \CCC$,
we say a face $f$ is \emph{convex} if for all edges $\vec{e} \in \del f$,
we have $\vphi_{\vec{e}} \in [0,\pi/2)$.
We say $f$ is \emph{thin} if exactly two edges $\vec{e}, \vec{e'} \in \del f$
have nonzero $\vphi_{\cdot}$, and furtheremore,
	$0 < \vphi_{\vec{e}} = \pi - \vphi_{\vec{e'}} < \pi/2$.
\end{definition}

TODO figures for convex, thin faces

\begin{definition}
Given an extended circle pattern $c\in \CCC$, an edge $e$ is \emph{short}
if it has length 0, $l_{\vec{e}} = l_{\cev{e}} = 0$;
it is \emph{long} otherwise.
\end{definition}


\begin{definition}
Given an extended circle pattern $c\in \CCC$, a \emph{thick path}
is a sequence of faces $f_0,f_1,\ldots,f_n$
such that $f_i,f_{i+1}$ share a long edge.
A \emph{thick cycle} is a thick path with $f_0 = f_n$.
\end{definition}



%===============================================================================

\section{Main results??}


Our general strategry for obtaining a (non-singular) circle pattern
is to start with an assignment $\underline{\theta} = (\theta_\cdot)$
for which we know an extended circle pattern exists
(say from results of \ocite{BS}),
and a path $\gamma$ in $\TTT$ starting at $\underline{\theta}$.
We then attempt to lift $\gamma$ to a path $\tilde{\gamma}$ in $\CCC$,
so that $\tilde{\gamma}$ remains (face) non-singular
(vertex non-singularity is already determined by $\gamma$).

Note that since $\sum_{e\in E} \theta_e
= 2\pi |E| - \sum_{\vec{e} \in \vec{E}} \vphi_{\vec{e}}
= 2\pi |E| - \sum_{f\in F} \Phi_f$,
we see that maintaining face non-singularity of $\tilde{\gamma}$
forces $\sum \theta_e$ to be constant.
We will show that this is the only obstruction on $\gamma$
to the lifting to such $\tilde{\gamma}$.

To that end, let $L$ be the $(|E|-1)$-plane distribution on $\TTT$
tangent to the level sets of $\sum \theta_v$.
The following proposition proves it(?) up to first order at a point:


\begin{proposition}
\label{p:point_lift}
Let $c \in \CCC$ be an extended circle pattern such that
\begin{itemize}
	\item $\Phi_f = 2\pi$ for all $f\in F$;
	\item all faces are eithe convex or thin;
	\item every pair of faces is connected by a thick path.
\end{itemize}
Let $K_c = \ker (A_*|_c : T_c \CCC \to T_{A(c)} \PPP)$.
Then $B_*(K_c) = L_{B(c)}$.

In other words, for any vector $b = b^e \del_{\theta_e}$
with sum of coefficients 0, one can vary $c$ so that its
first order change in $\theta_\cdot$ is $b$,
and also remains face non-singular up to first order.
\end{proposition}


\begin{lemma}
Let $c \in \CCC$ be as in \prpref{p:point_lift}.
Then $A$ is a submersion in a neighbourhood of $c$.
\end{lemma}

\begin{proposition}
Let $c\in \CCC$ be as in \prpref{p:point_lift}.
TODO full lifting statement.
\end{proposition}

%===============================================================================

%\input{appendix}
\end{document}

