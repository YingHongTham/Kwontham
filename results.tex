\section{Main ?? results}


Our general strategry for obtaining a (non-singular) circle pattern
is to start with an assignment $\underline{\theta} = (\theta_\bullet)$
for which we know an extended circle pattern exists
(say from results of \ocite{BS}),
and a path $\gamma$ in $\TTT$ starting at $\underline{\theta}$.
We then attempt to lift $\gamma$ to a path $\tilde{\gamma}$ in $\CCC$,
so that $\tilde{\gamma}$ remains (face) non-singular
(vertex non-singularity is already determined by $\gamma$).

Note that since $\sum_{e\in E} \theta_e
= 2\pi |E| - \sum_{\vec{e} \in \vec{E}} \vphi_{\vec{e}}
= 2\pi |E| - \sum_{f\in F} \Phi_f$,
we see that maintaining face non-singularity of $\tilde{\gamma}$
forces $\sum \theta_e$ to be constant.
We will show that this is the only obstruction on $\gamma$
to the lifting to such $\tilde{\gamma}$.

To that end, let $L$ be the $(|E|-1)$-plane distribution on $\TTT$
tangent to the level sets of $\sum \theta_v$.
The following proposition proves it (avoid using `it'?) up to first order at a point:


\begin{proposition}
\label{p:point_lift}
Let $c \in \CCC$ be an extended circle pattern such that
\begin{itemize}
	\item $\Phi_f = 2\pi$ for all $f\in F$;
	\item all faces are eithe convex or thin,
		with at least one convex face;
	\item every pair of faces is connected by a thick path.
\end{itemize}
Let $K_c = \ker (\Phi_*|_c : T_c \CCC \to T_{\Phi(c)} \PPP)$.
Then $\Theta_*(K_c) = L_{\Theta(c)}$.

In other words, for any vector $b = b^e \del_{\theta_e}$
with sum of coefficients 0, one can vary $c$ so that its
first order change in $\theta_\bullet$ is $b$,
and also remains face non-singular up to first order.
\end{proposition}


Before we prove this, it is convenient to first prove the following:

\begin{lemma}
Let $c \in \CCC$ be as in \prpref{p:point_lift}.
Then $\Phi$ is a submersion in a neighbourhood of $c$.
\end{lemma}

\begin{proof}
We construct vectors $\beta_f \in T_c \CCC$ so that
$\Phi_*(\beta_f) = \ddd{\Phi_f}$.
The vector
\begin{equation}
\label{e:alpha_f}
\alpha_f := \ddd{r_f} - \sum_{\vec{e} \in \del f} 
	\frac{\tan \vphi_{\vec{e}}}{r_f} \ddd{\vphi_{\vec{e}}}
	\in T(\RRR \times \QQQ)
\end{equation}
doesn't change $l_{\vec{e}}$,
so is in $T_c \CCC$.
Intuitively, $\alpha_f$ is like
pulling the center of $C_f$ up off the plane, increasing $r_f$ and decreasing all
$\vphi$'s.
Its pushforward under $\Phi$ is simply
$\Phi_*(\beta_f) = \frac{1}{r_f} \sum_{\vec{e}\in \del f} \tan \vphi_{\vec{e}}
	\ddd{\Phi_f}$.


If $f$ is a convex face, then the $\tan \vphi_{\vec{e}}$ are all non-negative,
being 0 if and only if $e$ is short.
so $\Phi_*(\alpha_f)$ is a negative multiple of $\ddd{\Phi_f}$;
we choose $\beta$ so that
\begin{equation}
\label{e:beta_f_convex}
\beta_f := \beta \cdot \alpha_f; \;\;\;
\Phi_*(\beta_f) = \ddd{\Phi_f}
\end{equation}


If $f$ is thin, the coefficients sum to 0 so that $\Phi_*(\beta_f) = 0$,
so we need a different approach. First suppose $f$ shares an edge
$e$ with a convex face $f'$. We can increase $l_e$
by increasing $\vphi_{\vec{e}},\vphi_{\cev{e}}$
while holding $r_f,r_{f'}$ constant.
This will affect both $\Phi_f, \Phi_{f'}$,
so we use $\beta_{f'}$ to make $\Phi_{f'}$ constant.
More explicitly, consider
\begin{equation}
\label{e:alpha_e}
\alpha_e = \frac{1}{2 r_f \cos \vphi_{\vec{e}}} \ddd{\vphi_{\vec{e}}}
+ \frac{1}{2 r_{f'}\cos \vphi_{\cev{e}}} \ddd{\vphi_{\cev{e}}}
\end{equation}
Then we can take
\begin{equation}
\label{e:beta_f_thin}
\beta_f = r_f \cos \vphi_{\vec{e}}
	(\alpha_e - \frac{1}{r_{f'} \cos \vphi_{\cev{e}}} \beta_{f'})
\end{equation}
which increases $l_{\vec{e}}, l_{\cev{e}}$ at unit speed,
$dl_{\vec{e}}(\alpha_e) = dl_{\cev{e}}(\alpha_e) = 1$.
We can repeat this procedure with $f'$ set to this thin face,
and $f$ set to another thin face adjacent to it, etc.
\end{proof}

\begin{proof}[Proof of \prpref{p:point_lift}]
We construct a vector $u_e \in K_c \subseteq T_c \CCC$ for each long edge $e$
such that if $\Theta_*(u_e) = \sum_{e' \in E} a^{e'} \ddd{\theta_{e'}}$,
\begin{itemize}
	\item $a^e = 1$;
	\item $a^{e'} \leq 0$ for all $e' \neq e$;
	\item $a^{e'} = 0$ for short edges $e'$.
\end{itemize}

First suppose that $e$ is between two convex faces $f, f'$
(so $\vec{e} \in \del f, \cev{e} \in \del f'$).
Observe that $\alpha_e$ from \eqnref{e:alpha_e}
increases $\Phi_f$ and $\Phi_{f'}$,
and we can compensate using $\beta_f$ from \eqnref{e:beta_f_convex},
so we consider
\[
w_e = \alpha_e - \frac{1}{\cos \vphi_{\vec{e}}} \beta_f
	- \frac{1}{\cos \vphi_{\cev{e}}} \beta_{f'} \in K_c
\]
%Since $\theta_e = \pi - \vphi_{\vec{e}} - \vphi_{\cev{e}}$,
%we need to show that the coefficients of $\ddd{\vphi_{\vec{e}}},
%\ddd{\vphi_{\cev{e}}}$ sum to 0.
For $\vec{e'} \in \del f \cup \del f' \backslash \{\vec{e},\cev{e}\}$,
the $\ddd{\vphi_{\vec{e'}}}$ component only appears in $\beta_f$
or $\beta_{f'}$, which is positive by construciton
(this requires $f,f'$ to be convex faces);
thus, $d\theta_{e'}(w_e) \geq 0$ for such $e'$, and
is equal 0 if and only if $e'$ is short.
Finally, since $w_e \in K_c$, it leaves the sum $\sum_{e\in E} \theta_e$
constant, so $d\theta_e(w_e) < 0$, so we can take
\begin{equation}
\label{e:u_e}
u_e = (d\theta_e(w_e))^\inv \cdot w_e
\end{equation}

In general, we consider a thick path of faces $f_0, \ldots, f_n$,
such that all faces except $f_0$ and $f_n$ are thin.


\end{proof}


\begin{proposition}
Let $c\in \CCC$ be as in \prpref{p:point_lift}.
TODO full lifting statement.
\end{proposition}

%===============================================================================

