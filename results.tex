\section{Main ?? results}


TODO: make a list of images of vectors, e.g.
$\Theta_*(\ddd{\vphi_{\vec{e}}}) = -\ddd{\theta_e}$.


Our general strategry for obtaining a (non-singular) circle pattern
is to start with an assignment $\underline{\theta} = (\theta_\bullet)$
for which we know an extended circle pattern exists
(say from results of \ocite{BS}),
and a path $\gamma$ in $\TTT$ starting at $\underline{\theta}$.
We then attempt to lift $\gamma$ to a path $\tilde{\gamma}$ in $\CCC$,
so that $\tilde{\gamma}$ remains (face) non-singular
(vertex non-singularity is already determined by $\gamma$).

Note that since $\sum_{e\in E} \theta_e
= 2\pi |E| - \sum_{\vec{e} \in \vec{E}} \vphi_{\vec{e}}
= 2\pi |E| - \sum_{f\in F} \Phi_f$,
we see that maintaining face non-singularity of $\tilde{\gamma}$
forces $\sum \theta_e$ to be constant.
We will show that this is the only obstruction on $\gamma$
to the lifting to such $\tilde{\gamma}$.

To that end, let $L$ be the $(|E|-1)$-plane distribution on $\TTT$
tangent to the level sets of $\sum \theta_v$.
The following proposition proves it (avoid using `it'?) up to first order at a point:


\begin{proposition}
\label{p:point_lift}
Let $c \in \CCC$ be an extended circle pattern such that
\begin{itemize}
	\item $\Phi_f = 2\pi$ for all $f\in F$;
	\item all faces are eithe convex or thin,
		with at least one convex face;
	\item every pair of faces is connected by a thick path.
\end{itemize}
Let $K_c = \ker (\Phi_*|_c : T_c \CCC \to T_{\Phi(c)} \PPP)$.
Then $\Theta_*(K_c) = L_{\Theta(c)}$.

In other words, for any vector $a = a^e \del_{\theta_e}$
with sum of coefficients 0, one can vary $c$ so that its
first order change in $\theta_\bullet$ is $a$,
and also remains face non-singular up to first order.
\end{proposition}


Before we prove this, it is convenient to first prove the following:

\begin{lemma}
Let $c \in \CCC$ be as in \prpref{p:point_lift}.
Then $\Phi$ is a submersion in a neighbourhood of $c$.
\end{lemma}

\begin{proof}
We construct vectors $\beta_f \in T_c \CCC$ so that
$\Phi_*(\beta_f) = \ddd{\Phi_f}$.
The vector
\begin{equation}
\label{e:alpha_f}
\alpha_f := \ddd{r_f} - \sum_{\vec{e} \in \del f} 
	\frac{\tan \vphi_{\vec{e}}}{r_f} \ddd{\vphi_{\vec{e}}}
	\in T(\RRR \times \QQQ)
\end{equation}
doesn't change $l_{\vec{e}}$,
so is in $T_c \CCC$.
Intuitively, $\alpha_f$ is like
pulling the center of $C_f$ up off the plane, increasing $r_f$ and decreasing all
$\vphi$'s.
Its pushforward under $\Phi$ is simply
$\Phi_*(\beta_f) = \frac{1}{r_f} \sum_{\vec{e}\in \del f} \tan \vphi_{\vec{e}}
	\ddd{\Phi_f}$.


If $f$ is a convex face, then the $\tan \vphi_{\vec{e}}$ are all non-negative,
being 0 if and only if $e$ is short.
so $\Phi_*(\alpha_f)$ is a negative multiple of $\ddd{\Phi_f}$;
we choose $\beta$ so that
\begin{equation}
\label{e:beta_f_convex}
\beta_f := \beta \cdot \alpha_f; \;\;\;
\Phi_*(\beta_f) = \ddd{\Phi_f}
\end{equation}


If $f$ is thin, we actually have that $\alpha_f \in K_c$,
since the coefficients sum to 0 so that $\Phi_*(\alpha_f) = 0$,
so we need a different approach. First suppose $f$ shares an edge
$e$ with a convex face $f'$. We can increase $l_e$
by increasing $\vphi_{\vec{e}},\vphi_{\cev{e}}$
while holding $r_f,r_{f'}$ constant.
This will affect both $\Phi_f, \Phi_{f'}$,
so we use $\beta_{f'}$ to make $\Phi_{f'}$ constant.
More explicitly, consider
\begin{equation}
\label{e:alpha_e}
\alpha_e = \frac{1}{2 r_f \cos \vphi_{\vec{e}}} \ddd{\vphi_{\vec{e}}}
+ \frac{1}{2 r_{f'}\cos \vphi_{\cev{e}}} \ddd{\vphi_{\cev{e}}}
\end{equation}
Then we can take
\begin{equation}
\label{e:beta_f_thin}
\beta_f = r_f \cos \vphi_{\vec{e}}
	(\alpha_e - \frac{1}{r_{f'} \cos \vphi_{\cev{e}}} \beta_{f'})
\end{equation}
which increases $l_{\vec{e}}, l_{\cev{e}}$ at unit speed,
$dl_{\vec{e}}(\alpha_e) = dl_{\cev{e}}(\alpha_e) = 1$.
We can repeat this procedure with $f'$ set to this thin face,
and $f$ set to another thin face adjacent to it, etc.
\end{proof}

\begin{proof}[Proof of \prpref{p:point_lift}]
We construct a vector $u_e \in K_c \subseteq T_c \CCC$ for each edge $e$
and show that $\{\Theta_*(u_e)\}_{e\in E}$ spans an
$|E|-1$ dimensional space,
thus must be equal $L_{\Theta_c}$.


For long edges $e$, $u_e$ will have the following property:
if $\Theta_*(u_e) = \sum_{e' \in E} a^{e'} \ddd{\theta_{e'}}$, then
\begin{itemize}
	\item $a^e = 1$;
	\item $a^{e'} \leq 0$ for all $e' \neq e$;
	\item $a^{e'} = 0$ for short edges $e'$.
\end{itemize}
Furthermore, these $u_e$'s collectively satisfy the following
connectivity property:
consider the graph $G_\llong$ whose vertex set is the set of long edges,
and we connect two long edges $e,e'$ by an edge if there is some $e"$ such that
$a^e, a^{e'}$ are both nonzero in $\Theta_*(u_{e"})$;
then $G_\llong$ is connected.

In addition, for short edges $e$,
$u_e$ have the following property: again writing
$\Theta_*(u_e) = \sum_{e' \in E} a^{e'} \ddd{\theta_{e'}}$,
one has
\begin{itemize}
	\item $a^e = 1$;
	\item $a^{e'} = 0$ for other short edges $e'$.
\end{itemize}


Let us first suppose we have constructed such $u_e$,
and show that these properties ensure that
$\{\Theta_*(u_e)\}_{e\in E}$ spans an $|E|-1$ dimensional space.
This is a simple exercise in linear algebra,
but we show it for completeness.


Put the $\Theta_*(u_e)$'s into a $E \times E$ matrix,
denoted $M$,
so that the $e$-th row corresponds to $u_e$.
By virtue of $u_e \in K_c$, we have that $(1\; 1\; \cdots \; 1)^T$
is in the null space of $M$;
our goal is to show that it spans the null space.


Suppose $b = (b_e)^T$ is in the null space of $M$.
Reorder $E$, if necessary, so that long edges come before
short edges.
Then by the last property of $u_e$'s for long edges,
the top right block of long-by-short entries are 0.
Thus the vector $b_\llong := (b_e)_{e \in \llong}^T$ is in the null space
of the submatrix $M_\llong$ of long-by-long entries.


Let $|b_e|$ be the largest among components of $b_\llong$;
rescale $b$ so that $b_e = 1$.
The $e$-th component of $M_\llong b_\llong$
is $1 - \sum a^{e'} b_{e'}$.
Since $\sum a^{e'} = 1$, this can 0 if and only if
for all $e'$ with $a^{e'} < 0$, we have exactly $b_{e'} = 1$.
By connectedness of $G_\llong$, this implies
$b_\llong = (1 \; \cdots \; 1)^T$.


Returning to the full $M$, it is now easy to see that
for a short edge $e$,
by looking at the $e$-th row of $M$,
we have $b_e = \sum a^{e'} = 1$,
so we are done.

%\footnote{
%This is a well-known argument/result about graph Laplacians:
%the first two properties of $u_e$'s ensure that
%$-M_\llong^T$ is essentially a graph Laplacian for $G_\llong$,
%so has rank = (\#vertices) - (\#components).}


Now we construct the $u_e$'s.
First suppose that $e$ is a long edge between two convex faces $f, f'$
(so $\vec{e} \in \del f, \cev{e} \in \del f'$).
Observe that $\alpha_e$ from \eqnref{e:alpha_e}
increases $\Phi_f$ and $\Phi_{f'}$,
and we can compensate using $\beta_f$ from \eqnref{e:beta_f_convex},
so we consider
\[
w_e = \alpha_e - \frac{1}{\cos \vphi_{\vec{e}}} \beta_f
	- \frac{1}{\cos \vphi_{\cev{e}}} \beta_{f'} \in K_c
\]
%Since $\theta_e = \pi - \vphi_{\vec{e}} - \vphi_{\cev{e}}$,
%we need to show that the coefficients of $\ddd{\vphi_{\vec{e}}},
%\ddd{\vphi_{\cev{e}}}$ sum to 0.
For $\vec{e'} \in \del f \cup \del f' \backslash \{\vec{e},\cev{e}\}$,
the $\ddd{\vphi_{\vec{e'}}}$ component only appears in $\beta_f$
or $\beta_{f'}$, which is positive by construciton
(this requires $f,f'$ to be convex faces);
thus, $d\theta_{e'}(w_e) \geq 0$ for such $e'$, and
is equal 0 if and only if $e'$ is short.
Finally, since $w_e \in K_c$, it leaves the sum $\sum_{e\in E} \theta_e$
constant, so $d\theta_e(w_e) < 0$, so we can take
\begin{equation}
\label{e:u_e}
u_e := (d\theta_e(w_e))^\inv \cdot w_e
\end{equation}

In general, we consider a thick path of faces $f_0, \ldots, f_n$,
such that all faces except $f_0$ and $f_n$ are thin,
and $f_i$ are distinct except possibly $f_0 = f_n$.
\footnote{Picture a slinky linking two rocks.}
Let $e_i$ be the long edge between $f_i$ and $f_{i+1}$,
and orient so that $\vec{e_i} \in \del f_i$, $\cev{e_i} \in \del f_{i+1}$.


We will construct $u_{e_i}$ for $i = 1,\ldots,n$.


We want to increase the lengths of $e_i$'s all together.
Let $E' = \{\vec{e_i}, \cev{e_i} | i=1,\ldots, n\}$.
Consider
\[
x :=
\sum_{i=0,\ldots,n-1} \alpha_{e_i}
=
\sum_{\vec{e} \in E'}
\frac{1}{2 r_{f_{\vec{e}}} \cos \vphi_{\vec{e}}} \ddd{\vphi_{\vec{e}}}
%\sum_{\substack{i = 0,\ldots, n-1 \\ \vec{e} = \vec{e_i}, \cev{e_i} }}
%\frac{1}{2 r_{f_0} \cos \vphi_{\vec{e_0}}} \ddd{\vphi_{\vec{e_0}}}
%+
%\frac{1}{2 r_{f_n} \cos \vphi_{\cev{e_{n-1}}}} \ddd{\vphi_{\cev{e_{n-1}}}}
%+
\]
with $\alpha_e$ from \eqnref{e:alpha_e}.
It is clear that $x$ increases all oriented $e_i$ lengths
$l_{\vec{e}}$ equally fast,
i.e. $dl_{\vec{e_i}}(x) = dl_{\cev{e_i}}(x) = 1$ for all $i$,
so $x\in T_c \CCC$.
It is also clear that for the thin faces $f_i$,
$x$ makes one of the angles $\vphi_{\cev{e_{i-1}}}, \vphi_{\vec{e_i}}$
increase and makes the other decrease, both at the same rate,
thus $d\Phi_{f_i}(x) = 0$.

The vector $x$ increase $\Phi_{f_0}$ and $\Phi_{f_n}$, so it is compensated
using $\beta_f$'s, so that
\[
y := x - \frac{1}{r_{f_0} \cos \vphi_{\vec{e_0}}} \beta_{f_0}
- \frac{1}{r_{f_n} \cos \vphi_{\cev{e_{n-1}}}} \beta_{f_n}
\in K_c
\]
Before we modify $y$ to give the desired $u_{e_i}$,
we note that for $\vec{e} \in (\cup \del f_i) \backslash E'$,
the coefficient of $\ddd{\vphi_{\vec{e}}}$ in $y$ is non-positive,
and is 0 if and only if $e$ is short;
so $\Theta_*(y)$ satisfies two of the three properties
desired in $u_e$.


Recall that for thin face $f$,
$\alpha_f$ from \eqnref{e:alpha_f} is in $K_c$,
and we have
\[
\Theta_*(\frac{r_{f_i}}{\tan \vphi_{\vec{e_i}}} \alpha_{f_i}) = 
	\ddd{\theta_{e_i}} - \ddd{\theta_{e_{i-1}}}
\]
Write $\bar{\alpha}_{f_i} = \frac{r_{f_i}}{\tan \vphi_{\vec{e_i}}} \alpha_{f_i}$.
Then it is easy to see that using $\bar{\alpha}_{f_i}$'s, we can
push the coefficients of $\ddd{\theta_{e_i}}$'s in $\Theta_*(y)$ around
so only one is nonzero.
More precisely, if
$\Theta_*(y) = \sum_e y^e \ddd{\theta_e}$, the image under $\Theta_*$
of the vector
\[
\bar{u}_{e_0} := y - y^{e_{n-1}} \bar{\alpha}_{f_{n-1}}
- (y^{e_{n-2}} + y^{e_{n-1}}) \bar{\alpha}_{f_{n-2}}
- \ldots
\]
would have $\ddd{\theta_{e_i}}$ coefficients 0 for all $i$ except 0.
Finally, we can take
\[
u_{e_0} := (d\theta_{e_0}(\bar{u}_{e_0}))^\inv \cdot \bar{u}_{e_0};
\;\;\;
u_{e_i} = u_{e_{i-1}} + \bar{\alpha}_{f_i}
\]

We still need to show the collective connectivity property of the $u_e$'s.
It is easy to see that the long edges of a convex face
are connected in $G_\llong$
(for example, $u_{e_0}$ defined above clearly does it for
$f_0$).
Furthermore, all the $\Theta_*(u_{e_i})$ share most of the same coefficients,
so $e_i$ are in the same connected component of $G_\llong$.
Thus connectivity of $G_\llong$ follows from the
thick-path-connectivity of faces.


Finally, for short edges $e$, we simply define $u_e$
as in \eqnref{e:u_e}
(here we do not worry about whether faces are thin or convex,
as there is no non-positivity condition of $a^{e'}$ is absent).
\end{proof}


\begin{proposition}
Let $c\in \CCC$ be as in \prpref{p:point_lift}.
Given a path $\gamma: [0,\veps] \to \TTT$
starting at $\gamma(0) = \Theta(c)$
and has constant value $(\sum \theta_e) \circ \gamma$,
there exists a lift $\tilde{\gamma} :[0, \veps] \to \CCC$
that starts at $\tilde{\gamma}(0) = c$
and has constant value $\Phi \circ \tilde{\gamma}$.
\end{proposition}


\begin{proof}
TODO.
\end{proof}

%===============================================================================

