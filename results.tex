\section{Main results??}


Our general strategry for obtaining a (non-singular) circle pattern
is to start with an assignment $\underline{\theta} = (\theta_\bullet)$
for which we know an extended circle pattern exists
(say from results of \ocite{BS}),
and a path $\gamma$ in $\TTT$ starting at $\underline{\theta}$.
We then attempt to lift $\gamma$ to a path $\tilde{\gamma}$ in $\CCC$,
so that $\tilde{\gamma}$ remains (face) non-singular
(vertex non-singularity is already determined by $\gamma$).

Note that since $\sum_{e\in E} \theta_e
= 2\pi |E| - \sum_{\vec{e} \in \vec{E}} \vphi_{\vec{e}}
= 2\pi |E| - \sum_{f\in F} \Phi_f$,
we see that maintaining face non-singularity of $\tilde{\gamma}$
forces $\sum \theta_e$ to be constant.
We will show that this is the only obstruction on $\gamma$
to the lifting to such $\tilde{\gamma}$.

To that end, let $L$ be the $(|E|-1)$-plane distribution on $\TTT$
tangent to the level sets of $\sum \theta_v$.
The following proposition proves it(?) up to first order at a point:


\begin{proposition}
\label{p:point_lift}
Let $c \in \CCC$ be an extended circle pattern such that
\begin{itemize}
	\item $\Phi_f = 2\pi$ for all $f\in F$;
	\item all faces are eithe convex or thin,
		with at least one convex face;
	\item every pair of faces is connected by a thick path.
\end{itemize}
Let $K_c = \ker (A_*|_c : T_c \CCC \to T_{A(c)} \PPP)$.
Then $B_*(K_c) = L_{B(c)}$.

In other words, for any vector $b = b^e \del_{\theta_e}$
with sum of coefficients 0, one can vary $c$ so that its
first order change in $\theta_\bullet$ is $b$,
and also remains face non-singular up to first order.
\end{proposition}


Before we prove this, it is convenient to first prove the following:

\begin{lemma}
Let $c \in \CCC$ be as in \prpref{p:point_lift}.
Then $A$ is a submersion in a neighbourhood of $c$.
\end{lemma}

\begin{proof}
We construct vectors $\beta_f \in T_c \CCC$ so that
$A_*(\beta_f) = \ddd{\Phi_f}$.
The vector
\begin{equation}
\beta_f = [ \ddd{r_f} - \sum_{\vec{e} \in \del f} 
	\frac{\tan \vphi_{\vec{e}}}{r_f} \ddd{\vphi_{\vec{e}}}] \cdot \alpha
\end{equation}
doesn't change $l_{\vec{e}}$,
so is in $T_c \CCC$
($\alpha$ to be determined).
Its pushforward under $A$ is simply
$A_*(\beta_f) = \frac{1}{r_f} \sum_{\vec{e}\in \del f} \tan \vphi_{\vec{e}}
	\ddd{\Phi_f}$.


If $f$ is a convex face, then the $\tan \vphi_{\vec{e}}$ are all positive,
so $A_*(\beta_f)$ is a negative multiple of $\ddd{\Phi_f}$;
we choose $\alpha$ so that $A_*(\beta_f) = \ddd{\Phi_f}$.
Intuitively, this is like
pulling the center of $C_f$ up off the plane, increasing $r_f$ and decreasing all
$\vphi$'s.


If $f$ is thin, the coefficients sum to 0 so that $A_*(\beta_f) = 0$,
so we need a different approach. First suppose $f$ shares an edge
$e$ with a convex face $f'$. We can increase $l_e$
by increasing $\vphi_{\vec{e}},\vphi_{\cev{e}}$
while holding $r_f,r_{f'}$ constant.
This will affect both $\Phi_f, \Phi_{f'}$,
so we use $\beta_{f'}$ to make $\Phi_{f'}$ constant.
More explicitly, consider
$\beta_e = \frac{1}{2\cos \vphi_{\vec{e}}} \ddd{\vphi_{\vec{e}}}
+ \frac{1}{2\cos \vphi_{\cev{e}}} \ddd{\vphi_{\cev{e}}}$.
Then we can take
$\beta_f = \cos \vphi_{\vec{e}}
	(\beta_e - \frac{1}{\cos \vphi_{\cev{e}}} \beta_{f'})$.
We can repeat this procedure with $f'$ set to $f$.
\end{proof}

\begin{proof}[Proof of \prpref{p:point_lift}]
proof
\end{proof}


\begin{proposition}
Let $c\in \CCC$ be as in \prpref{p:point_lift}.
TODO full lifting statement.
\end{proposition}

%===============================================================================

