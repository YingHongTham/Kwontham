\section{Main results??}


Our general strategry for obtaining a (non-singular) circle pattern
is to start with an assignment $\underline{\theta} = (\theta_\bullet)$
for which we know an extended circle pattern exists
(say from results of \ocite{BS}),
and a path $\gamma$ in $\TTT$ starting at $\underline{\theta}$.
We then attempt to lift $\gamma$ to a path $\tilde{\gamma}$ in $\CCC$,
so that $\tilde{\gamma}$ remains (face) non-singular
(vertex non-singularity is already determined by $\gamma$).

Note that since $\sum_{e\in E} \theta_e
= 2\pi |E| - \sum_{\vec{e} \in \vec{E}} \vphi_{\vec{e}}
= 2\pi |E| - \sum_{f\in F} \Phi_f$,
we see that maintaining face non-singularity of $\tilde{\gamma}$
forces $\sum \theta_e$ to be constant.
We will show that this is the only obstruction on $\gamma$
to the lifting to such $\tilde{\gamma}$.

To that end, let $L$ be the $(|E|-1)$-plane distribution on $\TTT$
tangent to the level sets of $\sum \theta_v$.
The following proposition proves it(?) up to first order at a point:


\begin{proposition}
\label{p:point_lift}
Let $c \in \CCC$ be an extended circle pattern such that
\begin{itemize}
	\item $\Phi_f = 2\pi$ for all $f\in F$;
	\item all faces are eithe convex or thin;
	\item every pair of faces is connected by a thick path.
\end{itemize}
Let $K_c = \ker (A_*|_c : T_c \CCC \to T_{A(c)} \PPP)$.
Then $B_*(K_c) = L_{B(c)}$.

In other words, for any vector $b = b^e \del_{\theta_e}$
with sum of coefficients 0, one can vary $c$ so that its
first order change in $\theta_\bullet$ is $b$,
and also remains face non-singular up to first order.
\end{proposition}


\begin{lemma}
Let $c \in \CCC$ be as in \prpref{p:point_lift}.
Then $A$ is a submersion in a neighbourhood of $c$.
\end{lemma}

\begin{proposition}
Let $c\in \CCC$ be as in \prpref{p:point_lift}.
TODO full lifting statement.
\end{proposition}

%===============================================================================

